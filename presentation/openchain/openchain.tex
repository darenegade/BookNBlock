\documentclass[17pt]{beamer}

\usepackage{default}

% Datum
\usepackage[ddmmyyyy]{datetime}
\renewcommand{\dateseparator}{.}

% Themenwahl
\usepackage[utf8]{inputenc}
\usetheme{Warsaw}  

\title{OpenChain}
\author{Daniel Sikeler}
\date{\today}

\begin{document}
\maketitle

\begin{frame}
\frametitle{Architektur}
\begin{itemize}
	\item nur \textbf{eine} Validierungsinstanz \\ $\Rightarrow$ Client-Server Architektur
	\item Observer/Anwender besitzen eine \textbf{readonly} Kopie
	\item Clients/Wallets verbinden sich mit dem Validator
	\item Smart Contracts
	\item Gateways
\end{itemize}
\end{frame}

\begin{frame}
\frametitle{Architektur}
\begin{itemize}
	\item Jede Organization hat eine separate Blockchain
	\item OpenChains können miteinander verbunden werden
	\item Anbindung an Bitcoin Blockchain möglich
\end{itemize}
\end{frame}

\begin{frame}
\frametitle{Vorteile}
\begin{itemize}
	\item keine Miner $\Rightarrow$ Transaktionen sind kostenlos und schnell
	\item einheitliche API (HTTP)
	\item kein zentrales Ledger, sondern viele spezialisierte
	\item keine Kryptowährung
\end{itemize}
\end{frame}

\begin{frame}
\frametitle{Aktuelle Entwicklung}
\begin{itemize}
	\item letzte Aktivität auf Github vor einem Jahr
	\item Server in C\#, Docker-Container vorhanden
	\item Referenz-Client in Javascript
	\item gute Dokumentation
\end{itemize}
\end{frame}
	
\begin{frame}
	\frametitle{Links}
	\begin{itemize}
		\item \href{https://www.openchain.org/}{Homepage}
		\item \href{https://github.com/openchain}{Github}
		\item \href{https://wallet.openchain.org/\#/addendpoint}{Endpoint anlegen}
	\end{itemize}
\end{frame}

\end{document}
